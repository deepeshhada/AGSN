\documentclass{article}

\usepackage{amsmath, amssymb}
\DeclareMathOperator{\E}{\mathbb{E}}
\usepackage{microtype}
\usepackage{graphicx}
\usepackage{subfigure}
\usepackage{booktabs} % for professional tables
\usepackage{hyperref}
\newcommand{\theHalgorithm}{\arabic{algorithm}}
\usepackage[accepted]{icml2020}
\icmltitlerunning{Zero-Shot Learning for Image Classification}

\begin{document}

\twocolumn[
	\icmltitle{Zero-Shot Learning for Image Classification}

	\begin{icmlauthorlist}
	\icmlauthor{Deepesh V. Hada}{affil1}
	\icmlauthor{Jayant Priyadarshi}{affil2}
	\icmlauthor{Kavita Wagh}{affil2}
	\end{icmlauthorlist}

	\icmlaffiliation{affil1}{Department of CSA, Indian Institute of Science, Bangalore, India}
	\icmlaffiliation{affil2}{Department of EECS, Indian Institute of Science, Bangalore, India}

	\icmlcorrespondingauthor{Deepesh V. Hada}{deepeshhada@iisc.ac.in}
	\icmlcorrespondingauthor{Jayant Priyadarshi}{jayantp@iisc.ac.in}
	\icmlcorrespondingauthor{Kavita Wagh}{kavitawagh@iisc.ac.in}

	\vskip 0.3in
]
\printAffiliationsAndNotice{}


\begin{abstract}
We consider the problem of zero-shot recognition: learning a visual classifier for a category with zero training examples, 
just using some embedding to represent the category and its relationship to the other categories, for which visual data are provided. 
The key to dealing with the unfamiliar or novel category is to transfer knowledge obtained from familiar classes to describe the unfamiliar class. 
Zero-shot learning is a promising learning method, in which the classes covered by the training instances and the classes we aim to classify are disjoint. 
It refers to a specific use case of machine learning (and therefore, deep learning) where we want the model to classify data based on very few or even no labeled examples, 
which means classifying data on the fly.
\end{abstract}


\section{Introduction}
In recent years, deep learning has achieved state-of-the-art performance across a wide range of computer vision tasks such as image classification. 
However, these deep learning methods rely on enormous amount of labeled data which is scarce for dynamically emerging objects. 
Practically, it is unrealistic to annotate everything around us, thus, making the conventional object classification methods infeasible. 

We focus on the extreme case when there is no labeled data, \textit{i.e.}, Zero-shot learning (ZSL), 
where the task is to recognize the novel categories' instances that have no labeled data available for training. 
ZSL assumes that the semantic label embeddings of both the seen and unseen classes are known apriori. 
ZSL, thus, learns to identify unseen classes by leveraging the semantic relationship between seen and unseen classes. 
It is important to note that the training and test classes are \textbf{disjoint} in ZSL.

%% citation from SABR
ZSL algorithms generally project the seen and unseen class instances onto a latent space that is expected to be robust for learning unseen class labels. 
This latent space is shared by both seen and unseen categories and is aimed to be a mapping between images and their class embeddings. 
The learning is carried out with the help of semantic descriptors associated with each class. 
These approaches, thus, tend to learn a latent space that is aligned towards the semantic class embedding \citep{sabr10, sabr12, sabr14, sabr17, sabr19, sabr23}, on which the input data is transformed. 
This transformation of data onto the latent space, however, makes these ZSL approaches suffer from the \textbf{hubness problem}: 
the transformed data vectors become \textit{hubs} for the nearby semantic class embeddings, leading to performance deterioration \citep{sabr18, sabr16, sabr9}. 
To mitigate the hubness problem, 
other approaches \citep{sabr29, sabr5, sabr31, sabr18, akanksha} learn a latent space for recognizing the seen class labels by aligning the semantic class embeddings towards this latent space. 

%% citations from CADA-VAE
There exists an orthogonal approach, wherein artificial images \citep{cada23} are generated to augment the training data. 
These synthetic images are usually generated from conditional generative models like conditional WGANs \citep{sabr25}. 
\citet{sabr25} improved image classification considerably, but having GAN-based loss functions, suffer from instability in training. 
Recent developments have then been made along this approach using Conditional VAEs (CVAEs). 
\citet{edgar} proposed an aligned VAE model, CADA-VAE, which 
learns a latent embedding of image features and class embedding via aligned VAEs optimized with cross-alignment and distribution-alignment objectives, 
and subsequently trains a classifier on sampled latent features of seen and unseen classes. 
This model has achieved a significant performance boost in many datasets, including state-of-the-art results in the AWA and AWA2 datasets.

%% citations from KG-GCN
There has been another very different approach for zero-shot recognition using the semantic class embeddings and a \textbf{Knowledge Graph}  (KG)
that encodes the relationship of the novel category to some familiar categories. 
The first paradigm is to use implicit knowledge representations, \textit{i.e.}, semantic embeddings, and hence, this is the same as the first set of approaches.
The alternative paradigm for zero-shot learning in these approaches is to use explicit knowledge bases or KGs, 
wherein one explicitly represents the knowledge as rules or relationships between objects. 
These relationships can then be used to learn zero-shot classifiers for new categories \citep{gcn33}. 
There has been a combination of the two such that the model distills both the implicit knowledge representations (i.e. semantic embedding) and 
explicit relationships (i.e. knowledge graph) for learning \textbf{Graph Convolutional Network} (GCN) based visual classifiers of novel classes \citep{abhinav}.

Also, irrespective of the latent space for transforming the data, there is an inherent \textbf{bias} in the model towards seen classes, 
which is referred to as the bias problem in literature. 
Due to this bias, the models generally perform poorly on unseen classes, and the predictions are primarily made in favour of the seen classes.

After considering these various approaches of GZSL for the task of image classification, 
we have implemented and modified \textbf{Semantically Aligned Bias Reducing} (SABR) \textit{Generalized Inductive} ZSL \citep{akanksha}, 
that gets rid of the hubness problem and the bias to a large extent and outperforms all the other models on \textbf{CUB} \citep{cub}, a fine-grained, medium scaled dataset.
The modification in the model comes in the form of Spectral Normalization \citep{spectralnorm}. 


\section{Related Work}
\label{related work}
We discuss the state-of-the-art performing models in this section. 
The datasets which are predominantly used for training and evaluating ZSL models are \textbf{CUB} \citep{cub}, \textbf{SUN} \citep{sun}, \textbf{AwA} \citep{awa}, \textbf{AwA2} \citep{awa2}  and \textbf{aPY} \citep{apy}.

% cite from abhinav
\citet{abhinav} take a different approach by constructing a new knowledge graph based on the Never-Ending Language Learning (NELL) for embeddings \citep{gcn3} and Never-Ending Image Learning (NEIL) for images \citep{gcn8} datasets and WordNet for word embeddings and ImageNet for images.

Since they evaluate their performances only on these datasets, their work cannot be compared with the other other models which evaluate on the aforementioned datasets.

A significant progress in ZSL was achieved by \citep{edgar}. 
Their approach takes feature generation one step further through a model (CADA-VAE)
where a shared latent space of image features and class embeddings is learned through \textit{modality-specific} aligned variational autoencoders, 
on which a SoftMax classifier is trained. CADA-VAE has achieved state-of-the-art results on \textit{AwA} and \textit{AwA2}.

\citet{changhu} proposed a model, GDAN, consisting of three components that perform GZSL, where 
the \textit{Generator} represents the semantic $\rightarrow$ visual methods; 
the \textit{Regressor} represents the visual $\rightarrow$ semantic methods; and 
the \textit{Discriminator} represents metric learning. The \textit{Regressor} is trained simultaneously with the CVAE using the cyclic consistency loss.
GDAN has pulled off state-of-the-art results on the \textit{SUN} dataset. 


\section{Task}
\label{task}
On the basis of data available during the training phase, ZSL can also be divided into two categories: \textbf{Inductive} and \textbf{Transductive} ZSL. 
In inductive ZSL \citep{sabr10, sabr14, sabr12, sabr17, sabr23, sabr29, sabr3, sabr25}, we are provided with the labeled seen class instances and the semantic embedding of unseen class labels during training. 
In transductive ZSL \citep{sabr21, sabr28}, in addition to the labeled seen class data and the semantic embedding of all labels, 
we are also provided with the \textbf{unlabeled instances} of unseen classes data during the training time.

Here, we assume that we do NOT have any unlabeled inputs corresponding to the unseen classes, and hence, we'll be dealing with \textbf{Inductive} ZSL.

Further, we consider a more practical and realistic version of ZSL, the \textbf{Generalized Zero-shot learning} (GZSL) problem. 
In the \textbf{classical} ZSL, data emerges only from unseen classes at test time, 
\textit{i.e.}, the performance of the algorithmic approach is solely judged on its classification accuracy on the novel unseen classes. 
In contrast, GZSL aims at maximizing performance on both seen and unseen classes, where the data during testing comes from both seen and unseen classes.

The task of GZSL is defined as follows:
\begin{equation}
S = \{(x, y, c(y))~|~x \in X,~ y \in Y^s, ~c(y) \in C\}
\end{equation}
is a set of training examples, consisting of image-features $x$, extracted by a CNN (ResNet-101), 
class labels $y$ available during training and semantic class embeddings $c(y)$. 
The class-embeddings are $312$-dimensional hand-annotated attribute vectors. Note that unlike the Transductive ZSL, no auxiliary training set is provided.

With conventional ZSL, the aim of the classifier during test time is to map,
\begin{equation}
X \to Y^u
\end{equation}
However, in this work, we focus on the more realistic and challenging setup of GZSL where the aim is to learn a classifier that maps: 
\begin{equation}
X \to Y^u \cup Y^s
\end{equation}


\section{Model}
\label{model}
TODO

\section{Experiments and Results}
\label{experiments}
TODO

\medskip

\section*{Software and Data}
The repository can be found \href{https://github.com/deepeshhada/SABR}{here} and has been implemented in Python 3.7.4 using PyTorch.

\medskip

\bibliography{final_report}
\bibliographystyle{icml2020}
\end{document}